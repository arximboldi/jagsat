% 
% PROJECT PLAN DOCUMENT 
% =====================
% 
% Time-stamp: <2009-11-20 01:45:55 raskolnikov>
% (c) 2009 The JAGSAT development team.
% 

\documentclass[12pt,a4paper]{article}

\usepackage{raskolnikov}

\title{\large JAGSAT project\\\huge Project Status Report 1}
\author{
  Juan Pedro Bolívar Puente\\ 
  Aksel Junkkila\\
  Guillem Medina\\ 
  Sarah Lindstrom\\ 
  Alberto Villegas Erce\\ 
  Thomas Forss
}
\location{Project Course \\ \textit{Åbo Akademy}}
% \date{}

\begin{document}
\maketitle

\tableofcontents
\pagebreak

\section{Time tracking}

\begin{table}[h!]
\small
\begin{tabular}{ l | r | r | r | r | r | r}
Time tracking	&J-P	&Alberto	&Thomas	&Guillem	&Aksel	&Sarah\\\hline\hline
Management	&5	&5		&5		&5		&5		&5\\
Lectures		&4	&12		&12		&12		&12		&12\\
Meetings		&16	&16		&18		&16		&16		&16\\
Group Learning&7	&4		&5		&5		&5		&4\\
Design		&20	&12		&8		&4		&10		&3\\
Coding		&30	&0		&10		&15		&4		&0\\
Testing		&10	&2		&0		&0		&2		&0\\
Documentation&4	&5		&3		&0		&5		&8\\
Self Learning	&4	&6		&10		&8		&4		&2\\
Artwork		&0	&0		&8		&0		&5		&0\\
Total			&100&62		&79		&65		&68		&50
\end{tabular}
\caption{Time track}
\label{tab:timetrack}
\end{table}

\section{Completed work}

\begin{table}[h!]
\small
\begin{tabular}{ l | r | r}
Task					&Status	&Completed at\\\hline\hline
Project Plan (1)			&90 \%	&02.10.2009\\
Project Vision (1)		&90 \%	&02.10.2009\\
Venture Cup 1			&100 \%	&11.11.2009\\
Design Document		&100 \%	&11.12.2009\\
Prototype source code	&100 \%	&15.12.2009\\
Risk map ver. 1			&100 \%	&11.12.2009\\
Project Plan (2)			&100 \%	&11.12.2009\\
Project Vision (2)		&100 \%	&11.12.2009\\
\end{tabular}
\caption{Complete work}
\label{tab:compwork}
\end{table}

\section{Planned work}

The following tasks table has been defined as a guide for the project development. \ref{tab:tasks}

\begin{table}[h!]
\small
\begin{tabular}{ l | c | r | c}
Task				&Status		&Est. workload left (h)	&Deadline \\\hline\hline
Project Plan (2) 	&90	\%		&4					&14.12.2009\\
Project Vision (2)	&90 	\%		&4					&14.12.2009\\
Alpha release		&60 	\%		&220				&15.01.2010\\
\ - Core layer		&			&40					&\\
\ - In game menu	&			&20					&\\
\ - Main menu		&			&40					&\\
\ - Player menu		&			&10					&\\
\ - Map loading		&			&20					&\\
\ - Map component	&			&30					&\\
\ - Setup stage		&			&20					&\\
\ - Game stage		&			&20					&\\
\ - Integration		&			&20					&\\
Beta release		&10 \%		&100				&12.02.2010\\
\ - Load/Save game	&			&20					&\\
\ - Improving widgets	&			&30					&\\
\ - Art				&			&50					&\\
Venture Cup 2		&0 \%		&30					&11.03.2010\\
Beta release 2		&0 \%		&24[[[0				&12.03.2010\\
\ - Art				&			&40					&\\
\ - Programming	&			&200				&\\
Status report 2		&0 \%		&16					&12.03.2010\\
Risk Map ver. 2		&10 \%		&20					&12.03.2010\\
Graphics and sound	&0 \%		&12					&12.03.2010
\end{tabular}
\caption{Tasks}
\label{tab:tasks}
\end{table}

\section{Risks}

In the beginning of the project we identified the following risks.

\begin{itemize}
\item \textit{Availability of the resources. As the team members are working in the team besides other studies and activities, it is very difficult to guarantee the availability of the resources.}

We have kept meetings on a regular basis to keep each other motivated and up to date with the tasks we are doing. This has worked well. We have not experienced that other activities (trips, other courses) have hindered the project in any other way than we expected.
\item \textit{Dependency. This game depends on how the TribeFlame device works. Changes in the device may occur, and might make this project very sensitive.}

The device itself has not been a problem, since we have not yet been in need of testing on the device. Software has proven to be a challenge for the most of us.

\item \textit{Motivation. The team members are not motivated enough and may not finish the project on time.}

No problems with the motivation. Everyone is motivated and eager to get the game working.

\item \textit{Lack of knowledge. The programming language Python, which is new to a few team members, may prove to be more challenging than we thought in the beginning of the project.}

Python has indeed been a challenge for many of the team members. We have had two separate meetings concerning Python to make sure everyone has basic knowledge of the language. Other than problems learning we have had some problems regarding the API sent by the client, these problems include not being able to understand how it works properly as well as some features not providing the functionality that they were meant to do.

\item \textit{Over ambitious. The team may plan a game which is not possible to program within the timeframe of the project.}

We have reviewed our project tasks several times. Today we have a clear view of the game and are confident that we can finish it in time. We have a few functions we can skip if we think we do not have the time to finish the game.

\item \textit{Time table not estimated correctly. The schedule is not realistic and has to be revised during the project.}

We are on schedule and have been able to work according to the schedule the whole time so far. We try to get something done every week and keep each other updated through e-mail.

\item \textit{Injury, illness or other issues that prevent people from working. If this happens we will have to adjust the schedule and switch roles from back-up to main.}

No such things have occurred.
\end{itemize}

\section{Status report summary}

The project is proceeding according to plan. The design phase of the game is well documented and we have had meetings where we have discussed how the game should work. We have had deep group- brainstorming meetings to come up with an effective design, both in terms of playability and usability. This has been very rewarding and has allowed us to come up with a sound game design and a detailed definition of the user interface that takes usability as a main goal and not as a postponed feature.

The new programming language Python has proven to be quite the challenge for some of us, which came as no surprise. We have also had meetings concerning Python to get everyone onboard and up-to-date. The lack of documentation in the API we are using (TFlib) has also created some problems. We have had meetings with the client were we have discussed the game and the device requirements. The team work with TribeFlame is going well and we can rely on them helping us out with technical information etc.

We communicate very much through e-mail, despite the regular meetings. This has worked well and this form of communication suits us. We have lately been focusing on creating a clear design for the game so that we do not have to redesign a lot in the future. We have done the base layers and are planning to implement the applications on top. As the design phase is now coming to an end, we are beginning to focus on the logic of the applications.

\end{document}
