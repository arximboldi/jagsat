% 
% PROJECT PLAN DOCUMENT 
% =====================
% 
% Time-stamp: <2010-03-26 05:17:14 raskolnikov>
% (c) 2009 The JAGSAT development team.
% 

\documentclass[12pt,a4paper]{article}

\usepackage{raskolnikov}

\title{\large JAGSAT project\\\huge Project Status Report 3}
\author{
  Juan Pedro Bolívar Puente\\ 
  Aksel Junkkila\\
  Guillem Medina\\ 
  Sarah Lindstrom\\ 
  Alberto Villegas Erce\\ 
  Thomas Forss
}
\location{Project Course \\ \textit{Åbo Akademy}}
% \date{}

\begin{document}
\maketitle

\begin{center}
\textbf{Revision history}

\begin{tabular}{ l | l | l | l }
Date			&Version	&Description		&Author\\\hline\hline
22.03.2010	&1.0		&Status Report		&Alberto
Villegas\\
25.03.2010	&1.1		&Status Report		&Juan Pedro Bolívar
\end{tabular}
\label{tab:rev}
\end{center}

\vfill
Copyright 2009 AUTHORS.
Permission is granted to copy, distribute and/or modify this document under the terms of the GNU Free Documentation License, Version 1.1 or any later version published by the Free Software Foundation;  with no Invariant Sections, with no Front-Cover Texts, and with no Back-Cover Texts. A copy of the license is included in the ``D9: Licenses''  document entitled `GNU Free Documentation License''.

\pagebreak
\tableofcontents
\pagebreak

\section{Time tracking}

\begin{table}[h!]
\small
\begin{tabular}{ l | r | r | r | r | r | r}
Time tracking    &J-P  &Alberto &Thomas &Guillem &Aksel &Sarah\\\hline\hline
Management       &5    &5       &2      &5       &4     &5\\
Lectures         &0    &0       &0      &0       &0     &0\\
Meetings         &5    &5       &5      &5       &5     &5\\
Group Learning   &0    &0       &0      &0       &0     &0\\
Design           &10   &12      &0      &4       &0     &0\\
Coding           &80   &15      &20     &47      &4     &0\\
Testing          &20   &8       &8      &0       &2     &0\\
Documentation    &4    &12      &3      &0       &0     &30\\
Self Learning    &10   &10      &5      &8       &10    &10\\
Artwork          &10   &0       &8      &0       &30    &0\\
Total            &145  &65      &51     &69      &55    &50
\end{tabular}
\caption{Time track}
\label{tab:timetrack}
\end{table}

\section{Completed work}

The completed work is shown in table \ref{tab:compwork}

\begin{table}[h!]
\small
\begin{center}
\begin{tabular}{ l | r | r}
Task					&Status	&Completed at\\\hline\hline
Project Plan (1)			&100 \%	&02.10.2009\\
Project Vision (1)		&100 \%	&02.10.2009\\
Venture Cup 1			&100 \%	&11.11.2009\\
Design Document		&100 \%	&11.12.2009\\
Prototype source code	&100 \%	&15.12.2009\\
Risk map ver. 1			&100 \%	&11.12.2009\\
Project Plan (2)			&100 \%	&11.12.2009\\
Project Vision (2)		&100 \%	&11.12.2009\\
Alpha release			&100 \%	&10.02.2010\\
\ - Core layer			&100 \%	&\\
\ - In game menu		&100 \%	&\\
\ - Main menu			&95 \%	&\\
\ - Player menu			&100 \%	&\\
\ - Map loading			&100 \%	&\\
\ - Map component		&100 \%	&\\
\ - Setup stage			&100 \%	&\\
\ - Game stage			&100 \%	&\\
\ - Integration			&100 \%	&\\
Project Plan (3) 		&100\%	&12.02.2009\\
Project Vision (3)		&100 \%	&12.02.2009\\
Beta release			&95 \%	&20.02.2010\\
\ - Load/Save game		&90 \%	&\\
\ - Improving widgets	&90 \%	&\\
\ - Art				&90 \%	&\\
Venture Cup 2			&100 \%	&11.03.2010\\
Beta release 2			&90 \%	&12.03.2010\\
\ - Art				&80 \%	&\\
\ - Programming		&100 \%	&\\
Status report 2			&100 \%	&12.03.2010\\
Risk Map ver. 2		&100 \%	&12.03.2010\\
Graphics and sound		&90 \%	&12.03.2010
\end{tabular}
\end{center}
\caption{Complete work}
\label{tab:compwork}
\end{table}


\section{Risks}

In the beginning of the project we identified the following risks.

\begin{itemize}
\item \textit{Availability of the resources. As the team members are working in the team besides other studies and activities, it is very difficult to guarantee the availability of the resources.}

We have kept meetings on a regular basis to keep each other motivated and up to date with the tasks we are doing. This has worked well. We have not experienced that other activities (trips, other courses) have hindered the project in any other way than we expected.
\item \textit{Dependency. This game depends on how the TribeFlame device works. Changes in the device may occur, and might make this project very sensitive.}

It has happened that the device did not have a multi-touch screen in
the end. However, we had taken this into account from the beginning and
our user interface model was easy to translate. The TF library however
has proven to be a challenge for the most of us.

\item \textit{Motivation. The team members are not motivated enough and may not finish the project on time.}

No problems with the motivation. Everyone is motivated and eager to get the game working.

\item \textit{Lack of knowledge. The programming language Python, which is new to a few team members, may prove to be more challenging than we thought in the beginning of the project.}

Python has indeed been a challenge for many of the team members. We have had two separate meetings concerning Python to make sure everyone has basic knowledge of the language. Other than problems learning we have had some problems regarding the API sent by the client, these problems include not being able to understand how it works properly as well as some features not providing the functionality that they were meant to do.

\item \textit{Over ambitious. The team may plan a game which is not possible to program within the timeframe of the project.}

We have reviewed our project tasks several times. Today we have a
clear view of the game and are confident that we can finish it in
time. We have a few functions we can skip if we think we do not have
the time to finish the game. Also, by skipping some very time
consuming functionality we could use more time in polishing and adding
nice-to-have features.

\item \textit{Time table not estimated correctly. The schedule is not realistic and has to be revised during the project.}

We have found some problems with the initial schedule. Some changes have been done according with, what we estimate. our resources at this moment.

\item \textit{Injury, illness or other issues that prevent people from working. If this happens we will have to adjust the schedule and switch roles from back-up to main.}

No such things have occurred.
\end{itemize}

\section{Status report summary}

The final stage has been a challenge for everyone of us. With the ICT-Showroom as death line we had to split ourselves in order to manage with our personal studies and the load of work to do. We really wanted to impress people and finally we succeeded. The crazy previous days leaded to a satisfying morning where everyone wanted to try our prototypes and ask about our project. Competence was hard and finally other projects won the prizes but we were really satisfied.

Work is not over. Our over ambitious initial specifications have
placed our project on a playable state but there are still some
functionalities to be add. The low coupling and predictability in our
design makes us possible to handle a product right now and several
upgrades in the future.

Still, the current result is very satisfying. The game is fully
playable, includes most of the planned features plus some extra
ones. The user interface is very innovative and sets a new ground for
exploration in the digital board game market.

Anyway, the Project Course experience has been great. The chance of
working on a group, for a company and in a whole project is unique and
we will all take an advance of it.

\end{document}
