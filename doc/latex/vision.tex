% 
% DESIGN DOCUMENT 
% ===============
% 
% Time-stamp: <2009-11-20 01:32:45 raskolnikov>
% (c) 2009 The JAGSAT development team.
% 

\documentclass[12pt,a4paper]{article}

\usepackage{raskolnikov}

\title{\large JAGSAT project\\\huge Vision Document}
\author{
  Juan Pedro Bolívar Puente\\ 
  Aksel Junkkila\\
  Guillem Medina\\ 
  Sarah Lindstrom\\ 
  Alberto Villegas Erce\\ 
  Thomas Forss
}
\location{Project Course \\ \textit{Åbo Akademy}}
% \date{}

\begin{document}

\maketitle

\begin{center}
\textbf{Revision history}

\begin{tabular}{ l | l | l | l }
Date			&Version	&Description		&Author\\\hline\hline
01.10.2009	&0.1		&Project Vision		&Thomas Forss\\
10.12.2009	&1.0		&Delivery editing	&Alberto Villegas 
\end{tabular}
\label{tab:rev}
\end{center}

\vfill
Copyright 2009 AUTHORS.
Permission is granted to copy, distribute and/or modify this document under the terms of the GNU Free Documentation License, Version 1.1 or any later version published by the Free Software Foundation;  with no Invariant Sections, with no Front-Cover Texts, and with no Back-Cover Texts. A copy of the license is included in the ``D9: Licenses''  document entitled `GNU Free Documentation License''.

\pagebreak
\tableofcontents
\pagebreak

\disabletodo

\section{Product description}

Our client is TribeFlame, a starting company, which are developing a
new gaming device for playing board games and other social
games. Their aim is to create a new device that allows people to
socialize during breaks or free time without having to carry around
huge and heavy board games and packages. Now that we are involved in a
technological revolution where lots of information can be stored in
small devices and following the steps of Amazons kindle and other
touch screen tablets, we are intrigued to be part of such an
innovative project.

Our project is to create a game for the gaming device that TribeFlame
is developing. The device will imitate board games, social and other
multi-player games through a computer with a multi-touch touch screen
as interface. Games will be downloadable through the TribeFlame online
store, which will work much like the mobile-phone stores (Appstore,
Ovi store). The games will be stored in the device which you can
easily carry with you wherever you go.

\section{Project description}

After the first discussion with our client we were given free hands to
develop the project as we see fit as long as it fits the
non-functional requirements, in other words the hardware and other
technical aspects. Our client gave us a few interesting ideas
regarding the project, which we are taking into consideration. We have
decided to develop a game that involves Risk as well as Tower defense.

\section{Functional requirements}

\subsection{Functional requirements for the game}

\begin{itemize}
\item The game must load different maps according with the map definition
  file.
\item The game must be divided in three phases: reinforcement, attack
  and movement.
\item Each phase must follow the rules defined for the gameplay.
\item The users must take turns one by one.
\item The users must choose to attack a new region or stay in current.
\item The users must only choose to move to a new region that is on the
  border to his current region.
\item The game should contain a help menu describing all the stages.
\item The game should be able save profiles of configuration.
\item The game should be able to be saved from any stage.
\item The game should show the regions where you can move.
\item The users could spend money on defense or attack depending on the
  amount of units he controls.
\end{itemize}

\begin{todo}[Alberto]
  Should we include more stuff here? Remember that the functional
  requirements has to be sort by priority in this order "must", "should",
  "could" and "won't".
\end{todo}

\section{Non-functional requirements}

\subsection{Non-functional requirements for the game}

\begin{itemize}
\item The game must be from 2 to 6 players.
\item The game must be developed in Python.
\item The game has to be developed using the API provided by the client.
\item The game should be playable after a 10 minutes tutorial.
\item The game must work on the TribeFlame device with an update rate
  of 50fps.
\end{itemize}

\subsection{Non-functional requirements for the Tribeflame device}

\begin{itemize}
\item 1024 x 768 pixels.
\item Touch screen with 2-touch. 
\item Processor 800 MHz ARM Cortex A8
\item Graphics HW-accelerated
\item OpenGL 2.0 ES.
\end{itemize}

\section{Development requirements}

\begin{itemize}
\item Weekly group meetings where we discuss accomplishments and problems.
\item Biweekly meetings with the client for feedback and updates on
  the platform as well as the project.
\item One person in charge and one or two backups/assistants for each
  part of the project.
\item Strictly following the timetable.
\item Communication through irc, email and possibly Skype.
\item Use of SVN for updating the project.
\item Use of development tools; version management and unit testing.
\item Use of the task management system and Subversion control system.
\end{itemize}

\end{document}