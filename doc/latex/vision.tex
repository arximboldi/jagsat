% 
% DESIGN DOCUMENT 
% ===============
% 
% Time-stamp: <2009-11-20 01:32:45 raskolnikov>
% (c) 2009 The JAGSAT development team.
% 

\documentclass[12pt,a4paper]{article}

\usepackage{raskolnikov}

\title{\large JAGSAT project\\\huge Vision Document}
\author{
  Juan Pedro Bolívar Puente\\ 
  Aksel Junkkila\\
  Guillem Medina\\ 
  Sarah Lindstrom\\ 
  Alberto Villegas Erce\\ 
  Thomas Forss
}
\location{Project Course \\ \textit{Åbo Akademy}}
% \date{}

\begin{document}

\maketitle

\tableofcontents
\pagebreak

\section{Product description}

Our client is TribeFlame, a starting company, which are developing a
new gaming device for playing board games and other social
games. Their aim is to create a new device that allows people to
socialize during breaks or free time without having to carry around
huge and heavy board games and packages. Now that we are involved in a
technological revolution where lots of information can be stored in
small devices and following the steps of Amazons kindle and other
touch screen tablets, we are intrigued to be part of such an
innovative project.

Our project is to create a game for the gaming device that TribeFlame
is developing. The device will imitate board games, social and other
multi-player games through a computer with a multi-touch touch screen
as interface. Games will be downloadable through the TribeFlame online
store, which will work much like the mobile-phone stores (Appstore,
Ovi store). The games will be stored in the device which you can
easily carry with you wherever you go.

\section{Project description}

After the first discussion with our client we were given free hands to
develop the project as we see fit as long as it fits the
non-functional requirements, in other words the hardware and other
technical aspects. Our client gave us a few interesting ideas
regarding the project, which we are taking into consideration. We have
decided to develop a game that involves Risk as well as Tower defense.

\section{Functional requirements}

\subsection{Functional requirements for the game}

\begin{itemize}
\item The users take turns one by one
\item A user can choose to attack a new region or stay in current
\item A user can only choose to move to a new region that is on the
  border to his current region
\item A user can spend money on defence or attack depending on the
  amount of units he controls
\end{itemize}

\subsection{Function requirements for the Tribeflame device}

Not available, we have not yet received details from our clients
regarding the device.

\begin{todo}[Juan Pedro]
  Can we fill this now?
\end{todo}

\section{Non-functional requirements}

\begin{itemize}
\item The game must be developed in Python.
\item The game has to be developed using the API provided by the client.
\item The game should be playable after playing the tutorial.
\item The game must work on the TribeFlame device.
\item The game should be a multi-player game.
\end{itemize}

\section{Development requirements}

\begin{itemize}
\item Weekly group meetings where we discuss accomplishments and problems.
\item Biweekly meetings with the client for feedback and updates on
  the platform as well as the project.
\item One person in charge and one or two backups/assistants for each
  part of the project.
\item Strictly following the timetable.
\item Communication through irc, email and possibly Skype.
\item Use of SVN for updating the project.
\item Use of development tools; version management and unit testing.
\item Use of the task management system and Subversion control system.
\end{itemize}

\end{document}